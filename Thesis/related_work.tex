\section{Related Work}
\label{sec:rel}

\subsection{Document representation}
Documents can be represented in multiple ways when used within text classification. One of these methods is bag-of-words, which means that each document is represented as a vector with the length of the vocabulary. Each entry in that vector represents how many times the corresponding word occurs within that document. Often, this representation is expanded upon with TF-IDF, which includes the inverse document frequency of words as well. Therefore the occurences of words within are scaled based on the total amount of documents in which that word occurs.\\
While bag-of-words has been the standard representation for many years, it does have significant disadvantages. This is foremost caused by the loss of word order within that representation. Also, words with similar meaning, such as "hate" and "loathe" are equally far apart within this representation as words with totally different meaning such as "hate" and "love". This means that this representation is less well equiped to deal with nuanced differences in meaning and context.\\

%Deze sectie bestaat uit een aantal "blokken", waarin je per blok de relevante literatuur beschrijft. 
%
%Neem alleen literatuur op die van belang is voor jouw onderzoeksvraag en deelvragen.
%
%Typisch heb je 1 blok voor je hoofdvraag en per deelvraag \textbf{RQi} een blok. 
%
%
%\subsection{RQ1}
%
%\subsection{RQ2}